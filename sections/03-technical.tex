% ----- 03 — Technical and Logistical Notes -----
% Structure: Weather, Health and safety, Tips, Know before you go, Arrival, Leave No Trace

\section{Technical and Logistical Notes}\label{sec:technical}

\subsection{Weather and Climate}

San Pedro de Atacama (2,400 m / 7,900 ft) enjoys mild March conditions:

\begin{itemize}[noitemsep]
  \item \textbf{Daytime Temperature:} 20--28°C (67--82°F)
  \item \textbf{Nighttime Temperature:} 8--9°C (46--48°F)
  \item \textbf{Precipitation:} Minimal (5--11 mm over 2--3 days)
  \item \textbf{Daylight:} 11--12 hours with intense sunshine
  \item \textbf{UV Index:} Extreme; can exceed 20 at altitude
\end{itemize}

\begin{safety}
  UV index can reach extreme levels (up to 16 in March, higher at altitude). Peak UV is typically 14:00--16:00; unprotected skin can burn in about 10 minutes. Sun protection (sunscreen SPF 30+, hat, long sleeves) is essential year-round.
\end{safety}

\subsection{Health and Safety}

\subsubsection{Altitude}

Altitude illness risk increases above 2,400 m (8,000 ft). \textbf{Acclimatize gradually:} limit elevation gain to 300--500 m/night above 3,000 m; include rest days; stay hydrated; avoid overexertion.

\textbf{Symptoms} (often 12--24 hours): headache, nausea, fatigue, dizziness, shortness of breath, sleep problems.

\textbf{Severe forms:}
\begin{itemize}[noitemsep]
  \item \textbf{HACE} (cerebral edema): confusion, loss of coordination
  \item \textbf{HAPE} (pulmonary edema): persistent cough, breathing difficulty
\end{itemize}

\textbf{If symptoms worsen, descend immediately.} Do not push through altitude illness.

\textbf{Diamox (Acetazolamide):} Prescription medication that can help prevent and reduce altitude sickness symptoms. Typical dosage is 125 mg twice daily, starting 1--2 days before ascent. Common side effects include increased urination, tingling in fingers/toes, and altered taste (especially carbonated drinks). Not suitable for people with sulfa allergies or certain medical conditions. \textbf{Get a prescription from Penn Student Health before your trip.}

\subsubsection{Thermoregulation}

Manage heat by staying hydrated and using sun protection (evaporation and radiation). At night or at altitude, cold is a factor. Use layering (wicking base, insulating mid, wind/waterproof shell) and avoid cotton. Day to night temperature swings can exceed 40°C at high elevation.

\begin{safety}
  Hypothermia and frostbite can occur in cold or windy conditions, even above freezing. Signs of hypothermia: shivering, confusion, drowsiness. Frostbite: numbness, then pain when rewarming. Move to shelter, remove wet clothing, warm slowly with dry layers and warm (non-alcoholic) fluids. Seek medical attention when appropriate.
\end{safety}

\subsubsection{Sun}

Dehydration and sunburn are common. Drink before you feel thirsty. Use broad-spectrum sunscreen (SPF 30+), reapply regularly, and wear a hat and long sleeves during peak UV hours.

\subsection{Tips for Staying Safe and Comfortable}

\begin{itemize}[noitemsep]
  \item Stay hydrated; drink before you're thirsty.
  \item Eat before you're hungry; carry snacks.
  \item Layer clothing; avoid cotton.
  \item Use sunscreen and a hat; cover head and wrists in cold.
  \item Allow time for acclimatization; do not push through altitude symptoms.
  \item Carry smaller banknotes in rural areas; change can be hard to get.
\end{itemize}

\subsection{Know Before You Go}

\textbf{Money.}
\begin{itemize}[noitemsep]
  \item Official currency: Chilean peso (CLP). Floating exchange rate; check current rate before travel (e.g.\ roughly 930--950 CLP = 1 USD; can move up to 10\% in a week).
  \item ATMs: REDBANC network, often 24/7 at malls, gas stations, grocery stores, banks. Avoid airport/hotel kiosks for exchange (poor rates).
  \item Cash useful in rural areas; cards (Visa, Mastercard, Amex, Diners Club) widely accepted in cities, less so in small towns.
  \item Tipping: about 10\% in restaurants; 15--20\% for excellent service. Tip in local currency (CLP).
\end{itemize}

\textbf{Adapters.}
\begin{itemize}[noitemsep]
  \item Voltage: 220V (230V), 50 Hz.
  \item Plug types: Type C (two round pins, ungrounded); Type L (three round pins in a line, grounded; most common). Bring a travel adapter for Type L.
  \item Most modern electronics accept 100--240V; check the rating on your device.
\end{itemize}

\begin{center}
  \includegraphics[width=0.7\linewidth]{assets/photos/plug_types.jpeg}\\[0.3em]
  \emph{Chilean plug types: Type C and Type L}
\end{center}

\textbf{Tap Water.}
\begin{itemize}[noitemsep]
  \item Tap water in San Pedro de Atacama is \textbf{not safe to drink}. Northern Chile's groundwater contains naturally occurring arsenic and may have mining-related contamination.
  \item Use bottled water for drinking and brushing teeth. Bottled water is widely available in stores and hotels.
  \item Boiling or filtering tap water does not remove arsenic; only bottled water is recommended.
\end{itemize}

\subsection{Arrival}

\textbf{Gateway:} El Loa Airport (CJC) in Calama → San Pedro de Atacama (100 km, 1h 15min transfer)

\textbf{Flight Requirements:}
\begin{itemize}[noitemsep]
  \item \textbf{Arrival:} Must arrive at CJC Airport by March 7, before or around 4:00 PM
  \item \textbf{Departure:} Can depart anytime after 10:00 AM on March 14
\end{itemize}

\subsection{Essential Camping Knots}

Knowing a few basic knots will help you set up camp, secure gear, and handle emergencies safely.

\begin{center}
  \includegraphics[width=0.48\linewidth]{assets/photos/square_knot.jpg}%
  \hfill%
  \includegraphics[width=0.48\linewidth]{assets/photos/bowline.jpg}\\[0.5em]
  \emph{Square Knot (left) and Bowline (right)}
\end{center}

\vspace{0.5em}

\textbf{Square Knot:} Joins two ropes of equal diameter. Not secure under heavy load. Use for bundling, first aid, or light tasks.

\textbf{Bowline:} Creates a fixed loop that won't slip. Ideal for rescue, securing gear, or tying around objects.

\begin{center}
  \includegraphics[width=0.48\linewidth]{assets/photos/taut_line_hitch.jpg}%
  \hfill%
  \includegraphics[width=0.48\linewidth]{assets/photos/clove_hitch.jpg}\\[0.5em]
  \emph{Taut-Line Hitch (left) and Clove Hitch (right)}
\end{center}

\vspace{0.5em}

\textbf{Taut-Line Hitch:} Adjustable knot for tent guy-lines. Slide to tighten or loosen; holds tension under load.

\textbf{Clove Hitch:} Quick knot for attaching rope to a post or tree. Easy to tie and untie, but can slip under varying load.

\begin{center}
  \includegraphics[width=0.48\linewidth]{assets/photos/truckers_hitch.jpg}%
  \hfill%
  \includegraphics[width=0.48\linewidth]{assets/photos/overhand_on_bite.jpg}\\[0.5em]
  \emph{Trucker's Hitch (left) and Overhand on a Bight (right)}
\end{center}

\vspace{0.5em}

\textbf{Trucker's Hitch:} Creates mechanical advantage (3:1) for tightening loads. Essential for securing tarps, gear, or packs.

\textbf{Overhand on a Bight:} Simple fixed loop in the middle of a rope. Useful for clipping carabiners or creating attachment points.

\subsection{Leave No Trace}

The seven principles of Leave No Trace provide minimum-impact practices for outdoor recreation:

\begin{enumerate}[noitemsep]
  \item \textbf{Plan Ahead and Prepare:} Know regulations and weather; prepare for emergencies; visit in small groups; repackage food to minimize waste; use maps instead of marking trails.
  \item \textbf{Travel and Camp on Durable Surfaces:} Use maintained trails and designated campsites; camp at least 200 feet (60 m) from water; concentrate use in popular areas, disperse use in pristine areas.
  \item \textbf{Dispose of Waste Properly:} Pack out all trash and food; use toilets when available; otherwise dig catholes 6 to 8 inches (15 to 20 cm) deep, at least 200 feet from water; pack out toilet paper.
  \item \textbf{Leave What You Find:} Do not remove natural or cultural objects; avoid building structures; do not introduce non-native species.
  \item \textbf{Minimize Campfire Impacts:} Use lightweight stoves for cooking; avoid campfires where they cause lasting damage.
  \item \textbf{Respect Wildlife:} Observe from a distance; do not feed or disturb animals.
  \item \textbf{Be Considerate of Others:} Respect other visitors; keep noise down; yield to other trail users; keep breaks and camps away from trails.
\end{enumerate}

\clearpage
