% ----- 06 — Provokers: Conflict Management and AAR (full text from sample) -----

\section{Conflict Management}\label{sec:conflict}

\begin{center}
\begin{tikzpicture}[node distance=2cm, auto]
  % Define colors
  \definecolor{step1}{RGB}{91,124,153}
  \definecolor{step2}{RGB}{109,120,92}
  \definecolor{step3}{RGB}{168,92,60}

  % Step 1
  \node[rectangle, rounded corners, draw=step1, fill=step1!20, text width=2.5cm, minimum height=1.5cm, align=center, thick] (step1) {\textbf{Step 1}\\Ventilation};

  % Step 2
  \node[rectangle, rounded corners, draw=step2, fill=step2!20, text width=2.5cm, minimum height=1.5cm, align=center, thick, right=of step1] (step2) {\textbf{Step 2}\\Owning \&\\Empathy};

  % Step 3
  \node[rectangle, rounded corners, draw=step3, fill=step3!20, text width=2.5cm, minimum height=1.5cm, align=center, thick, right=of step2] (step3) {\textbf{Step 3}\\Planning};

  % Arrows
  \draw[->, very thick, step1] (step1) -- (step2);
  \draw[->, very thick, step2] (step2) -- (step3);
\end{tikzpicture}
\end{center}

\vspace{1em}

\subsection{Step 1: Ventilation}

Explain your frustration to your friend, and, just as important, pause and listen to her perspective. Gathering all parties involved to express their concerns is a useful place to start managing a conflict. You can't resolve a fight if the people involved don't know each other's full story. As everyone speaks, you have the opportunity to articulate and put a name to your feelings, as well as understand the other person's perspective.

\textbf{Strategies for Ventilation:}
\begin{itemize}[noitemsep]
  \item Take turns; make sure everyone speaks
  \item Actively listen
  \item Paraphrase to make sure everyone is on the same page
  \item Expect to hear a different version of the situation from what you're experiencing
  \item Avoid sarcasm or bringing up cheap shots that aren't related to the conflict
  \item Accept the other person's point of view and feelings about the situation
\end{itemize}

\subsection{Step 2: Owning and Empathy}

After everyone has had a chance to express their concerns, acknowledge the ways you both contribute to the conflict. Each of you ``owns'' your actions. As you do that, you imagine what it's like to be the other person and empathize with the way they see the conflict.

\textbf{Strategies for Owning and Empathy:}
\begin{itemize}[noitemsep]
  \item Own what you believe you did or said; nothing more or less
  \item Take some time to set aside your own perspective and imagine the other person's experience
  \item Accept your contribution to the conflict
\end{itemize}

\subsection{Step 3: Planning}

Formulate action steps. Before moving to this step, make sure that everyone is done speaking and understands their role in the conflict. If you're still airing out your frustration, then you're not ready to make an action plan.

To make the plan, discuss what each person wants, expects from each other, and is willing to do to avoid the same conflict in the future. Follow-through is important here, so be realistic in your commitments and hold yourselves accountable to your agreement.

\textbf{Strategies for Planning:}
\begin{itemize}[noitemsep]
  \item State your expectations clearly
  \item Make sure you understand everyone else's expectations
  \item Expect that you will occasionally slip up with your new plan, and that's ok
  \item Accept that the other person has a choice whether he or she can meet your expectations
\end{itemize}

\section{How to Conduct an AAR}\label{sec:aar}

\subsection{What is an After Action Review?}

Generally, an AAR follows the following format, but Leaders of the Day (LODs) should feel free to develop their own format and activities to accomplish specific goals for the discussion.

\subsection{Format}

\begin{enumerate}
  \item Leaders (and/or the whole group) briefly retrace the route and day's activities, decisions, etc.\ (e.g., highs/lows).

  \item Leaders debrief topics such as things they did well and things they will do differently next time; if goals were accomplished; what they learned; other reflections on the day/decisions/conflicts, etc.

  \item Other group members provide feedback and input.
\end{enumerate}



\clearpage
