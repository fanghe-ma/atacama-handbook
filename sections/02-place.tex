% ----- 02 — Place: Background on Atacama -----
% Structure per OUTLINE.md: History, Geography, Animals and Plants, Astronomy Trivia

\section{Background on Atacama}\label{sec:place}

\subsection{History of Atacama}

\textbf{Ancient Settlement:} Human occupation dates to 13,000 years before present. Early hunters used the Altiplano above 3,600 m, following wild camelids in semi-nomadic patterns.

\textbf{Indigenous Empires:} The Tiwanaku (400--1100 A.D.) expanded from the Bolivian altiplano into northern Chile, leaving religious monuments and polychrome ceramics. The Inca incorporated the region in the 15th century, followed by Spanish conquest.

\textbf{Atacameño People:} Indigenous to the desert and altiplano, the Atacameño (Kunza, or Likanantaí) developed sophisticated water management around oases. Their language was suppressed under colonial rule. Today, roughly 30,000 Atacameño in Chile and 14,000 in Argentina co-manage protected areas like Reserva Nacional Los Flamencos.

\begin{center}
  \includegraphics[width=0.85\linewidth]{figures/photos/piedras_rojas.jpeg}\\
  \emph{Piedras Rojas (Red Stones) in the Atacama Desert}
\end{center}

\subsection{Geography of Atacama}

\textbf{The Driest Desert:} The Atacama runs 1,600 km along Chile's Pacific coast, covering 105,000--128,000 km². It's the driest non-polar desert on Earth (annual rainfall under 15 mm; some areas see no rain for decades). Aridity is caused by the cold Humboldt current, Pacific anticyclone, and double rain shadow from the Andes and Coast Range.

\textbf{Salar de Atacama:} Chile's largest salt flat (3,000 km²) sits 55 km south of San Pedro at 2,300 m elevation. This endorheic basin sees no outflow; evaporation exceeds 3,500 mm/year. Volcanic peaks rim the eastern edge: Licancabur (5,919 m), Acamarachi, and active Láscar volcano. The Altiplano features high plains, rift valleys, and dramatic slot canyons.

\begin{center}
  \includegraphics[width=0.85\linewidth]{figures/photos/san-pedro-de-atacama-map.jpg}\\
  \emph{Map of San Pedro de Atacama region and surrounding attractions}
\end{center}

\subsection{Animals and Plants of Atacama}

\subsubsection{Desert Flora}

\begin{center}
  \includegraphics[width=0.48\linewidth]{figures/photos/chanar.jpg}%
  \hfill%
  \includegraphics[width=0.48\linewidth]{figures/photos/rica_rica.jpg}\\[0.5em]
  \includegraphics[width=0.65\linewidth]{figures/photos/algarrobo.jpg}\\
  \emph{Desert trees and shrubs: Chañar, Rica Rica, and Algarrobo. Photos: Tierra Atacama}
\end{center}

\vspace{0.5em}

\textbf{Chañar Tree}
\begin{itemize}[noitemsep]
  \item Reaches 3--10 m tall with sharp thorns (drop before flowering)
  \item Intensely yellow flowers (September--October), edible fruit (November--January)
  \item Traditional uses: Arrope de Chañar (sweet syrup for throat/cough), Aloja de Chañar (fermented drink)
  \item Wood used for furniture, carpentry, charcoal
\end{itemize}

\textbf{Rica Rica Shrub}
\begin{itemize}[noitemsep]
  \item Endemic species, grows up to 1 m tall
  \item Aromatic greenish-yellow leaves, light-purple flowers (September)
  \item Traditional medicine: stomach, kidney, and circulatory problems
\end{itemize}

\textbf{Tamarugo \& Algarrobo} \emph{(Prosopis)}
\begin{itemize}[noitemsep]
  \item Deep-rooted trees adapted to extreme aridity
  \item Tamarugo survives on groundwater in salt flats
  \item Algarrobo (white and black varieties) produces seed pods
  \item Important for local wildlife and traditional uses
\end{itemize}

\textbf{Cacti \& High-Altitude Plants}
\begin{itemize}[noitemsep]
  \item \textbf{Copiapoa:} Endemic spherical cacti with yellow flowers (endangered)
  \item \textbf{Llareta:} High-altitude cushion plant (centuries old, slow-growing)
  \item \textbf{Tillandsia:} Air plants in coastal fog zones (lomas)
\end{itemize}

\textbf{Flowering Desert} \emph{(Desierto Florido)}
\begin{itemize}[noitemsep]
  \item Occurs in rare El Niño wet years
  \item Hundreds of endemic species bloom simultaneously
  \item Species include: añañuca, lion's claw, pata de guanaco
\end{itemize}

\subsubsection{Wildlife Viewing Guide}

\begin{center}
  \includegraphics[width=0.48\linewidth]{figures/photos/flamingos.jpg}%
  \hfill%
  \includegraphics[width=0.48\linewidth]{figures/photos/guanacos.jpg}\\[0.5em]
  \includegraphics[width=0.65\linewidth]{figures/photos/laguna.jpg}\\
  \emph{Atacama wildlife: flamingos, guanacos, and saline lagoons. Photos: GoChile}
\end{center}

\vspace{0.5em}

\textbf{Flamingos} (Three Species)
\begin{itemize}[noitemsep]
  \item Chilean, Andean, and James's flamingos
  \item Feed on diatoms and brine shrimp in mineral-rich waters
  \item Best viewing: Laguna Chaxa, Salar de Pujsa, Las Vegas de Quepiaco
\end{itemize}

\textbf{Vicuña} \emph{(Lama vicugna)}
\begin{itemize}[noitemsep]
  \item Small camelids (up to 65 kg) with fine wool
  \item Once endangered, now protected
  \item Best viewing: Salar de Tara, puna meadows (3,200--4,800 m)
\end{itemize}

\textbf{Guanaco} \emph{(Lama guanicoe)}
\begin{itemize}[noitemsep]
  \item Larger wild camelids (150 kg, 1.9 m tall)
  \item Gray faces, small ears, light brown to dark red fur
  \item Best viewing: Los Flamencos National Reserve
\end{itemize}

\textbf{Other Wildlife}
\begin{itemize}[noitemsep]
  \item \textbf{Culpeo fox:} South American grey fox, bushes and steppes
  \item \textbf{Vizcacha:} Nocturnal rodent at Salar de Tara
  \item \textbf{Horned Coot:} 3,000--5,200 m, autumn at Miscanti and Miñiques lagoons
  \item \textbf{Cougar:} Rare, follows guanaco populations throughout Los Flamencos
\end{itemize}

\subsection{Astronomy Trivia}

\begin{itemize}[noitemsep]
  \item \textbf{300+ clear nights per year:} April to October averages 90 to 98\% clear night probability
  \item \textbf{Some weather stations have never recorded rain:} One location went without significant rainfall from 1570 to 1971
  \item \textbf{Home to 70\%+ of the world's most advanced observatories:} Extreme aridity, high altitude, and near-zero light pollution create ideal conditions
  \item \textbf{ALMA operates 24/7/365:} 66 precision antennas spread up to 16 km apart at 5,000 m elevation
  \item \textbf{\$1.4 billion telescope:} ALMA is the most expensive ground-based telescope in operation
  \item \textbf{10× better resolution than Hubble:} In its wavelength range, ALMA achieves 10 times better detail
  \item \textbf{First black hole image (2019):} ALMA participated in the Event Horizon Telescope project that captured the first direct image of a black hole
  \item \textbf{Mars rover testing site:} NASA uses Atacama to test Mars rovers due to similar soil and climate conditions
  \item \textbf{Weekend tours available:} ALMA offers public tours on Saturdays and Sundays; book 3+ months ahead
\end{itemize}

\clearpage
