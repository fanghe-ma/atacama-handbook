% ----- 05 — Case Studies (full text from sample) -----

\section{Case Studies}\label{sec:cases}

\subsection{Lhotse 2006}

``So, the decision is yours,'' Jordan looks to his companions.

Rodrigo Jordan is the leader of a twelve-member Chilean expedition to Mt.\ Lhotse, the fourth highest mountain in the world. Often referred to as ``the forgotten mountain,'' Lhotse shares a Base Camp and most of the route with the southern face route of Everest. Most climbers visiting the region opt to attempt the highest mountain in the world rather than visiting Lhotse. This year is no different: there are 20 expeditions in base camp preparing for Everest with just 5 preparing for Lhotse, celebrating the 50th year since its first ascent.

The expedition was launched to celebrate the life and contributions of Jordan's mentor, Claudio Lucero, to mountaineering in Chile. Each climber on the team has been Lucero's student at one point over the years. Lucero was to become the oldest man to summit an 8,000-meter peak should he reach the summit. Lhotse will be the 6th 8,000-meter peak for Chile. Jordan has all but closed his company in Santiago, Chile during this time, as most of his key employees are accompanying him.

The team have worked well, and have fully equipped the mountain by the time an early weather window arrives. However, Lucero, an excellent mountaineer, has not been able to shake the Khumbu Cough he developed on arriving in the valley and is struggling to keep up the pace with the younger climbers. Jordan makes the difficult decision to exclude him from the summit attempt. The other climbers are all fit, and ready to go.

Jordan faces another difficult question. Traditional, non-alpinist summit pushes for large expeditions involve selecting a small summit team, who will be supported by various teams positioned along the route. The support teams help carry gear, break trail and are ready to help in an emergency, greatly increasing the summit team's chances of succeeding. The division of the team is usually based on a number of factors: strength, equipment, weather. However, looking at his team of strong, determined climbers, each one of them appears capable of making it to the summit. Jordan makes the unprecedented decision to allow the entire team to attempt the summit, breaking with tradition. It does not go unnoticed that if they succeed they will be one of the largest national teams to summit an 8,000-meter peak ever.

Logistical questions settled, the expedition team begins to move up the mountain in two smaller sub groups.

\subsubsection{A Falling Object}

On the second day of the push, as the first group, led by Jordan, move up from Camp 2 to Camp 3, the second group, under the leadership of Kiko Guzman, operations manager of Vertical, spot something falling from high on the Lhotse face, above Camp 4. They radio the information to the advance team. Both teams attempt to identify what fell. However, it is dusk and visibility is bad. No calls come over the radio. The teams are forced to give up and retreat into their tents to rest and hydrate.

The following day however, as the first group trudge up slowly behind a long line of Everest climbers, the object comes into sight. Lying just 25 meters or so below the route, it is obvious that it is in fact a climber. Knowing that he had fallen over 600 meters down the face, and had spent the night outside exposed to the elements at over 7,000 meters and from the long line of climbers walking past him, the team were sure he was dead. As they got closer however, they spotted movement and thus decided to check his condition.

One of the team's doctors, Sebastian, was climbing with this advanced group. Along with Jordan and the sherpas, Sebastian notes that the climber, a Czech alpinist they had met briefly a few days earlier, while still alive, is in a coma, and close to dying.

``So, it's your decision,'' as the long line of Everest climbers continues to walk past their little team, Jordan looks to Sebastian, and the other members of the group.

Sebastian is the younger of the team's two doctors, whose main role during the expedition has been to monitor the health and progress of Lucero, setting a ``doable'' pace for the older climber. With the decision to leave Lucero behind at Base Camp, Sebastian's chances of summiting have increased significantly. This would be his first 8,000-meter peak. He has taken significant time out of his medical residency over the past two years to train for two expeditions, one of which was cancelled before they even left Chile.

Along with Sebastian, is seasoned mountaineer, Ernesto Olivares. Ernesto, a professor of mountaineering, is one of Chile's strongest climbers. This is his third 8,000-meter peak and he is generally considered amongst the team to be the most experienced climber. Ernesto, however, is attempting to summit Lhotse without supplemental oxygen. He is also filming the high-altitude segments of the expedition.

The other climbers are Nico, his father, Nico Papa, Max and two sherpas. Nico Papa's company is one of the main expedition sponsors and should he summit on schedule, it will be on his 50th birthday. Summiting an 8,000 m has been a long standing dream of his, having been at junior school under the formidable headmaster and mountaineer, George Lowe. Nico chico, is the youngest of the team at 23, with very little climbing experience under his belt. Max is a strong experienced climber, and also the general manager of the team's gear sponsor.

The only way to try and save the Czech is to move him down the mountain. He has bled heavily from the head, and is suffering from frostbite in his hands and feet. To move him is going to take considerable energy. A tall order at 7,000 meters. Below, Kiko and his team listen over the radio to the discussion. Patiently waiting on their team mates, as a tricky tent sharing arrangement means that changes to the advance team's schedule will affect their own chances for a summit attempt.

\subsection{Questions}

\begin{enumerate}
  \item How should the team come to a decision about what to do?
  \item What responsibility does the team have to this climber, abandoned by his own climbing partner?
  \item At what level of effort or commitment can one discharge one's ethical responsibilities?
\end{enumerate}

\clearpage
